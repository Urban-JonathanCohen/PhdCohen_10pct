\section{Research Project}
\subsection{Specific Problem}
More specifically, the UN declares that the SDGs ‘are a blueprint to achieve a better and more sustainable future’, but the fact is that they a are just a monitor not a guide to achieve those goals. In order to meet the SDGs, we need to take a holistic approach and unpack the complexity of the different systems. \par

More specifically, to produce sustainable City-Regions, Urban and Regional Planning requires comprehension not only about: (i) the interactions within the components of a system and (ii) how different systems are interconnected, but also about the Role of Space and Structure.\par

Underneath, the running hypothesis of this research project is that location and the spatial structure matters. In order to plan sustainable city-regions the spatial distribution of human activities must be considered. Neglecting spatial patterns, is neglecting valuable information about how places work. 

\subsection{Research Aim of the Project}
Given the specific challenge and problem detailed above, the aim of this study is to explore the relationship between performance, location and how events inter relate.
In order to explore these relationships, the project goal will be to spatialize (capture the spatial structure) behind specific urban sub-systems. \par

By mapping the location of activities and events, it will be able to model and describe the specific properties of the systems will be studying. \par

By revealing the spatial dimension of urban systems, new and relevant information could be taken into consideration for Planning better City-Regions. 
