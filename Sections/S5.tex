\section{Case Study: Waste MGMT \& Food Sytems}
Based on personal preferences, relevance for creating sustainable settlements (SDG11 \& SDG12) and a potential knowledge gap in how space has been taken into account in the planning processes, two urban systems will be explored: Solid Waste Management and Food Systems are potential candidates to be explored as case studies for the present project. \par

In order to build a case towards the importance of spatialization and understanding the spatial characteristics of its this project will explore the food and waste management systems.   

\subsection{Waste MGMT System}
According to the Director of Social, Urban, Rural and Resilience Global Practice, Ede Iljazs-Vasquez, Solid Waste Management (MSWM) is at the core of building sustainable and resilient city-regions. He specifies that this is not just a matter of providing technical solutions; how to make waste collection routes more efficient or how increase the waste-to-energy transformation ratio. Waste Management must consider other social and environmental systems. \par

In the countries of the European Union, in 2017 almost half of the waste generated was recycled (46.6 \%) and on average, approximately 5\% of the renewable energy is generated by municipal waste. In this sense, some countries such as the Netherlands show the highest percentage (almost 30\%), evidencing the missed opportunity of understanding waste as a useful resource to generate energy. 


\subsection{Food System}
Food system and the waste of it are one of the important pieces of the waste management puzzle. Estimations suggest that 1/3 of the total food production is being lost or wasted. This means that 1/3 or even more of the efforts allocated in producing food are also being missed.  Consequently, research can help to reduce (i) food waste consumption at household level (ii) waste during the supply chain and/or (iii) the waste to energy processes inefficiencies. 


\subsection{Small note on the systems}
A systematic literature review on both systems is needed. Urban metabolism, Resource efficiency, how space is taken into consideration to understand these systems and how Urban and Regional Planning has been dealing with this waste mgmt and food is important for the future of this project. \par

It has become clear that that the synergies between land, waste and food management need to be understood properly and information about location and relationship can provide valuable information.  


